\documentclass[a4paper,10pt]{article}
\usepackage[english]{babel}
\usepackage[utf8x]{inputenc}
\usepackage[T1]{fontenc}
\usepackage{setspace}
%% Sets page size and margins
\usepackage[a4paper,top=3cm,bottom=2cm,left=3cm,right=3cm,marginparwidth=1.75cm]{geometry}
\usepackage{graphicx}
\usepackage{dcolumn}
\usepackage{booktabs}
\usepackage{longtable}
\usepackage{float}
\usepackage[toc,page]{appendix}
%% Useful packages

\usepackage{amsmath}
\usepackage{graphicx}
\graphicspath{ {/users/will/desktop/} }
\usepackage[colorinlistoftodos]{todonotes}
\usepackage[colorlinks=true, allcolors=blue]{hyperref}

\title{Workshop proposal}
\author{Will Labadie}

\doublespace

\begin{document}

\maketitle

%Placement of affordable housing in cities, gerrymandering, politician behavior.
%Use variation in governor party
% Do governors, through their political appointees, choose the location of affordable housing projects in any systematic way? To what extent are LIHTC's allocated according to political calculus? My sense is that affordable housing is occupied by mostly Democrat voters. So assume the governor is Republican: does affordable housing tend to locate in blue voting districts? Do affordable housing projects placed in red districts tend to get gerrymandered out?

\begin{abstract} 
This project seeks to determine whether or not affordable housing grants are allocated by partisan politicians according to a systematic political calculus. I use publicly available administrative data from the Low Income Housing Tax Credit (LIHTC) program, covering 1990-2010, to analyze the extent to which governors of U.S. states exert their political power over the location of some low-income voters. My results hold implications for understanding of politician and political appointee behavior and ethics, and the effectiveness of the LIHTC program as it stands today. 
\end{abstract}

\indent Low Income Housing Tax Credits (LIHTC's) are tax incentives distributed to states by the U.S. Department of Housing and Urban Development (U.S. HUD). They are meant to encourage "acquisition, rehabilitation, or new construction" of rental housing for low-income households (HUD, 2017). The LIHTC's, allocated to states at \$2.00 per capita, are then typically distributed to private investors by committees whose members are political appointees. In general, project developers will apply for the LIHTC through the state housing credit agency (the aforementioned committee of political appointees). If a project is approved, the owner of the development will agree to charge a rent below the market median in a certain number of units. The tax credit is granted to the project's investor, reducing the debt burden of developing the project. Figure 1 visualizes this process (Office of the Comptroller of the Currency, 2017).  \\
\indent The absolute control of governors and their political appointees in the allocation of LIHTC's presents them with an opportunity to shape the electoral power of the residents of those low-income housing projects. It is well-established that income and political affiliation are strongly correlated, though the political party that low income correlates with varies across states (Gelman 2011). Across the country, low-income individuals are much more likely to self-identify as Democrats than Republicans (Pew Research Center, 2016). This has been the case since at least 1990 (Vo, 2012). If low-income individuals have relatively predictable voting preferences and politicians have power over their location, it seems likely that a politician might exert that power. \\
\indent Governors of any type have an interest in expanding their political party's electoral power. Policy-motivated governors will benefit from their party's control of all branches of government; Governors motivated solely by re-election will be more likely to be re-elected if, all else held equal, the electoral map is friendlier to their party.  A Republican governor and his appointed state housing credit agency, for example, might choose to allocate LIHTC's to projects proposed in Democrat-controlled voting districts to maintain the strength of their current voting districts. Alternatively, if they are constrained by the number of LIHTC applicants, they may choose projects near Democrat-controlled voting districts and subsequently gerrymander the low-income housing out of their district. The governor and his committee would certainly not grant LIHTC's to projects located in swing districts. Politicians and political appointees might engage in distributive politics in allocating LITC's; this seems particularly likely among Democratic politicians given the average voting preferences of low-income Americans. \\
\indent This project analyzes the relationship between governor political party and location of projects granted LITC's. Does a state's governing political party influence the placement of affordable housing through LIHTC grants? 

\end{document}